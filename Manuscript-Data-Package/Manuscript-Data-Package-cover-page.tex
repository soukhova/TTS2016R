\documentclass[Royal,times,sageh]{sagej}

\usepackage{moreverb,url,natbib, multirow, tabularx}
\usepackage[colorlinks,bookmarksopen,bookmarksnumbered,citecolor=red,urlcolor=red]{hyperref}



% tightlist command for lists without linebreak
\providecommand{\tightlist}{%
  \setlength{\itemsep}{0pt}\setlength{\parskip}{0pt}}





\begin{document}


\setcitestyle{aysep={,}}

\title{TTS2016R: A dataset to study population and employment patterns
from the 2016 Transportation Tomorrow Survey (TTS) in the Greater
Toronto and Hamilton Area, Canada}

\runninghead{}

\author{Anastasia Soukhov\affilnum{}, Antonio Páez\affilnum{}}

\affiliation{\affilnum{}{}}



\begin{abstract}
This paper describes and visualises the data contained within the
\{TTS2016R\} data-package created in \texttt{R}, the statistical
computing and graphics language. In addition to a synthetic example,
\{TTS2016R\} contains home-to-work commute information for the Greater
Golden Horseshoe area in Canada retrieved from the 2016 Transportation
Tomorrow Survey (TTS). Included are all Traffic Analysis Zones (TAZ),
the number of people who are employed full-time per TAZ, the number of
jobs per TAZ, origin-destination trips, and calculated car travel time
from TAZ origin-destination centroid pairs. To illustrate how this
information can be analysed to understand patterns in commuting, we
estimate a distance-decay curve (i.e., impedance function) for the
region. \{TTS2016R\} can be freely downloaded and explored at:
\url{https://github.com/soukhova/TTS2016R} where the documentation and
code involved in data creation, manipulation, and the final data
products are detailed.
\end{abstract}

\keywords{Jobs; population; travel time; impedance; Greater Toronto and
Hamilton Area; Ontario, Canada; R}

\maketitle





\end{document}
