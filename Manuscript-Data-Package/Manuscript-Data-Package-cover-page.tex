\documentclass[Royal,times,sageh]{sagej}

\usepackage{moreverb,url,natbib, multirow, tabularx}
\usepackage[colorlinks,bookmarksopen,bookmarksnumbered,citecolor=red,urlcolor=red]{hyperref}



% tightlist command for lists without linebreak
\providecommand{\tightlist}{%
  \setlength{\itemsep}{0pt}\setlength{\parskip}{0pt}}





\begin{document}


\setcitestyle{aysep={,}}

\title{TTS2016R: A data set to study population and employment patterns
from the 2016 Transportation Tomorrow Survey (TTS) in the Greater Golden
Horseshoe Area, Ontario, Canada}

\runninghead{}

\author{Anastasia Soukhov\affilnum{}, Antonio Páez\affilnum{}}

\affiliation{\affilnum{}{}}



\begin{abstract}
This paper describes and visualises the data contained within the
\{TTS2016R\} data package created in \texttt{R}, the statistical
computing and graphics language. In addition to a synthetic example,
\{TTS2016R\} contains home-to-work commute information for the Greater
Golden Horseshoe (GGH) area in Canada retrieved from the 2016
Transportation Tomorrow Survey (TTS). Included are all Traffic Analysis
Zones (TAZ), the number of people who are employed full-time per TAZ,
the number of jobs per TAZ, the count of origin destination (OD) pairs
and trips by mode per origin TAZ, calculated car travel time from TAZ OD
centroid pairs, and associated spatial boundaries to link TAZ to the
Canadian Census. To illustrate how this information can be analysed to
understand patterns in commuting, we estimate a distance-decay curve
(i.e., impedance function) for the region. \{TTS2016R\} is a growing
open data product built on \texttt{R} infrastructure that allows for the
immediate access of home-to-work commuting data alongside complimentary
objects from different sources. The package will continue expanding with
additions by the authors and the community at-large by requests in the
future. \{TTS2016R\} can be freely explored and downloaded in the
associated \href{https://github.com/soukhova/TTS2016R}{Github
repository} where the documentation and code involved in data creation,
manipulation, and all open data products are detailed.
\end{abstract}

\keywords{Jobs; population; work; commute; travel time; impedance;
Greater Toronto and Hamilton Area; Greater Golden Horshoe Area, Ontario,
Canada; R}

\maketitle





\end{document}
