\documentclass[Royal,times,sageh]{sagej}

\usepackage{moreverb,url,natbib, multirow, tabularx}
\usepackage[colorlinks,bookmarksopen,bookmarksnumbered,citecolor=red,urlcolor=red]{hyperref}



% tightlist command for lists without linebreak
\providecommand{\tightlist}{%
  \setlength{\itemsep}{0pt}\setlength{\parskip}{0pt}}





\begin{document}


\setcitestyle{aysep={,}}

\title{TTS2016R: A data set to study population and employment patterns
from the 2016 Transportation Tomorrow Survey (TTS) in the Greater Golden
Horseshoe Area, Ontario, Canada}

\runninghead{}

\author{Anastasia Soukhov*\affilnum{1}, Antonio Páez\affilnum{1}}

\affiliation{\affilnum{1}{School of Earth, Environment and Society,
McMaster University, Hamilton, ON, L8S 4K1, Canada}}

\corrauth{Anastasia Soukhov,
\href{mailto:soukhoa@mcmaster.ca}{\nolinkurl{soukhoa@mcmaster.ca}}}


\begin{abstract}
This paper describes and visualises the data contained within the
\{TTS2016R\} data package created in \texttt{R}, the statistical
computing and graphics language. In addition to a synthetic example,
\{TTS2016R\} contains home-to-work commute information for the Greater
Golden Horseshoe (GGH) area in Canada retrieved from the 2016
Transportation Tomorrow Survey (TTS). Included are all Traffic Analysis
Zones (TAZ), the number of people who are employed full-time per TAZ,
the number of jobs per TAZ, the count of origin destination (OD) pairs
and trips by mode per origin TAZ, calculated car travel time from TAZ OD
centroid pairs, and associated spatial boundaries to link TAZ to the
Canadian Census. To illustrate how this information can be analysed to
understand patterns in commuting, we estimate a distance-decay curve
(i.e., impedance function) for the region. \{TTS2016R\} is a growing
open data product built on \texttt{R} infrastructure that allows for the
immediate access of home-to-work commuting data alongside complimentary
objects from different sources. The package will continue expanding with
additions by the authors and the community at-large by requests in the
future. \{TTS2016R\} can be freely explored and downloaded in the
associated \href{https://github.com/soukhova/TTS2016R}{Github
repository} where the documentation and code involved in data creation,
manipulation, and all open data products are detailed.
\end{abstract}

\keywords{Jobs; population; work; commute; travel time; impedance;
Greater Toronto and Hamilton Area; Greater Golden Horshoe Area, Ontario,
Canada; R}

\maketitle

\textbf{Anastasia Soukhov} is a PhD student in the School of Earth,
Environment \& Society at McMaster University. She has a masters and
bachelors in Civil Engineering with a transportation specialization. She
is enthusiastic about sustainable and equitable housing and
transportation and is a researcher within the Mobilizing Justice
Partnership working to establish data driven equity standards in
Canadian cities. Her recent work includes: studies on sustainable
passenger vehicle energy consumption and wheel-to-wheel emission
simulation, sustainable mobility policy optimization, and competitive
accessibility measures.

\textbf{Antonio Páez} is full professor in the School of Earth,
Environment \& Society at McMaster University. Recent work includes
studies on accessibility, spatial analysis of qualitative variables,
aging and mobility, transportation and social exclusion, the influence
of the built and social environments on travel behavior, social networks
and decision making, telework adoption, and healthcare provision and
trends. Paez has published widely, and is author or co-author of over
150 papers, many appearing in leading international journals. He
currently serves as Editor-in-Chief of the Journal of Geographical
Systems, and sits on the editorial boards of Transportation, Journal of
Transport Geography, Geographical Analysis, and International Journal of
Geographical Information Science, among others.



\end{document}
